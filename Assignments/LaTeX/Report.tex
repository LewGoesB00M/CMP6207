% -------------------------------------------------------------------------------
% Establish page structure & font.
\documentclass[12pt]{report}

\usepackage[total={6.5in, 9in},
	left=1in,
	right=1in,
	top=1in,
	bottom=1in,]{geometry} % Page structure

\usepackage{graphicx} % Required for inserting images
\graphicspath{{../.images/}} % Any additional images I use (BCU logo, etc) are from here.

\usepackage[utf8]{inputenc} % UTF-8 encoding
\usepackage[T1]{fontenc} % T1 font
\usepackage{float}  % Allows for floats to be positioned using [H], which correctly
                    % positions them relative to their location within my LaTeX code.
\usepackage{subcaption}
\usepackage{csquotes}

% -------------------------------------------------------------------------------
% Declare biblatex with custom Harvard BCU styling for referencing.
\usepackage[
    useprefix=true,
    maxcitenames=3,
    maxbibnames=99,
    style=authoryear,
    dashed=false, 
    natbib=true,
    url=false,
    backend=biber
]{biblatex}

\usepackage[british]{babel}

% Additional styling options to ensure Harvard referencing format.
\renewbibmacro*{volume+number+eid}{
    \printfield{volume}
    \setunit*{\addnbspace}
    \printfield{number}
    \setunit{\addcomma\space}
    \printfield{eid}}
\DeclareFieldFormat[article]{number}{\mkbibparens{#1}}

\addbibresource{Report.bib}

% -------------------------------------------------------------------------------
% To prevent "Chapter N" display for each chapter
\usepackage[compact]{titlesec}
\usepackage{wasysym}
\usepackage{import}

\titlespacing*{\chapter}{0pt}{-2cm}{0.5cm}
\titleformat{\chapter}[display]
{\normalfont\bfseries}{}{0pt}{\Huge}

% -------------------------------------------------------------------------------
% Custom macro to make an un-numbered footnote.

\newcommand\blfootnote[1]{
    \begingroup
    \renewcommand\thefootnote{}\footnote{#1}
    \addtocounter{footnote}{-1}
    \endgroup
}

% -------------------------------------------------------------------------------
% Fancy headers; used to show my name, BCU logo and current chapter for the page.
\usepackage{fancyhdr}
\usepackage{calc}
\pagestyle{fancy}

\setlength\headheight{37pt} % Set custom header height to fit the image.

\renewcommand{\chaptermark}[1]{%
    \markboth{#1}{}} % Include chapter name.


% Lewis Higgins - ID 22133848           [BCU LOGO]                [CHAPTER NAME]
\lhead{Lewis Higgins - ID 22133848~~~~~~~~~~~~~~~\includegraphics[width=1.75cm]{BCU}}
\fancyhead[R]{\leftmark}

% ------------------------------------------------------------------------------
% Used to add PDF hyperlinks for figures and the contents page.

\usepackage{hyperref}

\hypersetup{
    colorlinks=true,
    linkcolor=black,
    filecolor=magenta,
    urlcolor=blue,
    citecolor=black,
}

% ------------------------------------------------------------------------------
\usepackage{xcolor} 
\usepackage{colortbl}
\usepackage{longtable}
\usepackage{amssymb}
% ------------------------------------------------------------------------------
\usepackage{tcolorbox}
\newcommand{\para}{\vspace{7pt}\noindent}
% -------------------------------------------------------------------------------

\title{Untitled CMP6207 Report}
\author{Lewis Higgins - Student ID 22133848}
\date{March (?), 2025}

% -------------------------------------------------------------------------------

\begin{document}


\makeatletter
\begin{titlepage}
    \begin{center}
        \includegraphics[width=0.7\linewidth]{BCU}\\[4ex]
        {\huge \bfseries CMP6207 - Assignment 1}\\[2ex]
        {\large \bfseries  \@title}\\[50ex]
        {\@author}\\[2ex]
        {CMP6207 - Modern Data Stores}\\[2ex]
        {Module Coordinator: Konstantinos Vlachos}\\[5ex]
    \end{center}
\end{titlepage}
\makeatother
\thispagestyle{empty}
\newpage


% Page counter trick so that the contents page doesn't increment it.
\setcounter{page}{0}

\tableofcontents
\thispagestyle{empty}

\chapter*{Introduction}
\addcontentsline{toc}{chapter}{Introduction}
The report will be based on the design and implementation of a MongoDB NoSQL database system 
for the company "IoThings Home Automation Solutions", and its detailed brief containing the deadline
will become available during Week 2. There are two assignments in this module, where this report is
worth \textbf{60\%}, and a presentation (relating to code?) is the remaining \textbf{40\%}. This module 
will also incorporate elements of web design, with HTML and CSS also playing a role alongside the primary 
use of JavaScript.

\para \textbf{Work on this report is expected to fully begin in Week 4, or potentially even earlier within Week 2. 
A draft of this report may be due in Week 7.}

\chapter*{Module plan}
\begin{itemize}
    \item As minor occurrences happen, such as the installation of MongoDB, log them here.
    \item The timetabling of this module means that there's not much point trying to get ahead.
    \begin{itemize}
        \item This is partially a benefit, as it means you have time for CMP6228 and CMP6200.
    \end{itemize}
    \item (Currently, not much can be said here due to the brief not being available yet.)
    \item Kostas cares more for the functionality over aesthetics. While you should put effort into 
    the HTML and CSS parts, they're a lesser concern than the overall usability of the system.
\end{itemize}

\section*{Coding conventions}

\textbf{There are expected coding conventions within this module, though they are how you would 
code anyway. camelCase, spaces when assigning variables and between array contents ["Space", "Expected before me"].
Code must be indented to the relevant level which is likely handled by VSCode. Your lines should not exceed 
80 characters, and you may need to get a JS linter to enforce this. Your code should be commented.}

\vspace{10pt}

\subsubsection{Expected function format}

\begin{verbatim}
    function toCelsius(fahrenheit) {
        return (5 / 9) * (fahrenheit - 32);
    }
\end{verbatim}

\chapter{Database comparisons}
\begin{itemize}
    \item Compare types of databases here, ultimately ending on why MongoDB 
    is their best option.
\end{itemize}

\noindent Structured Query Language, or SQL, was developed by IBM following \textcite{codd_relational_1970}'s groundbreaking 
publication in the ACM journal, with the first commercial SQL implementation being published by Oracle in 1979 \autocite{oracle_history_nodate}.
SQL powers many relational database systems even today, though the problems associated with its age, most notably in 
the speed of its operations, are beginning to show in modern systems. Therefore, NoSQL ("Not Only SQL") was developed as 
a new form of database that stores data in a non-tabular, non-relational format which work to efficiently store 
semi-structured and unstructured data in a flexible, functional and scalable model for faster operations than standard 
relational databases in most scenarios \autocite{google_cloud_what_nodate, aws_what_nodate}. 

MongoDB, an acronym from "hu\textbf{mongo}us DB", aims to address some of SQL's key issues through its 

\chapter*{Conclusion}
\addcontentsline{toc}{chapter}{Conclusion}

Overall, something was done\dots


\printbibliography
\addcontentsline{toc}{chapter}{Bibliography}

\end{document}