\begingroup
\renewcommand\thechapter{A}
\titleformat{\chapter}[display]
{\normalfont\huge\bfseries}{}{20pt}{\Huge}
\setcounter{section}{0}
\setcounter{figure}{0} 

\chapter*{Appendix A - Data types}
\addcontentsline{toc}{chapter}{Appendix A - Data types}

The types of data stored in databases vary dependent on their file format. Section \ref{sec:DocDBs} referred to three 
major file formats used in document databases - JavaScript Object Notation (JSON), Binary JSON (BSON), and Extensible Markup Language (XML).
Each of these have their own distinct attributes, which will be thoroughly described in this appendix.

\section{JavaScript Object Notation (JSON)}
JSON files are\dots

% TODO: Why would I use JSON?

\section{Binary JSON (BSON)}
BSON files are\dots

% TODO: Why would I use BSON?

\section{Extensible Markup Language (XML)}
% TODO: Talk about what an XML file actually is.

XML files are not common in many databases in recent times. However, they do still have some benefits not found in JSON or
BSON, such as the creation of Document Type Definitions (DTDs) where data can be 
validated as it is retrieved by forming constraints such as what elements exist or what attributes an element can
or must have. DTDs do not allow for constraining data types (i.e. String, Integer, Float), which is a feature instead
provided by XML Schemas, an updated schema language that addresses DTD drawbacks through allowing not only type definitions
but also enforcing uniqueness and foreign key constraints as well as inheritance. Additionally, XML schemas are written
using XML syntax unlike DTDs. The only considerable drawback of XML schemas are the complexity of writing them, as they can
be considerably more complicated than DTDs.

% TODO: Show examples of DTDs & Schemas. See Lecture 4.
% TODO: Talk about XPath, XSLT and XQuery.

\endgroup