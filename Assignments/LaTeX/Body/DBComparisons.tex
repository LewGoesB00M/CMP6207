\chapter{Comparing NoSQL and Relational databases} 
% ! Chapter title doesn't fit the page.


% ? This section "shouldn't be too long, but enough to convince them."
% ! I'm targeting 1000 - 1250 words.
% ? This is NOT comparing *MongoDB* to relational databases.
% ? The brief mentions that it should be "generic just in case after seeing 
% ? your MongoDB database they decide to go with another software provider."

Industrial needs for fast and efficient data storage continue to grow exponentially year by year, and some of these needs fail to be met 
by typical relational database solutions. \textcite{corbelliniPersistingBigdataNoSQL2017} state that the types of data that require storage 
have changed since the first relational database systems of the 1970s, and as such, these more mature systems do not meet the requirements of
today's businesses with web-oriented systems storing magnitudes of unstructured data. The same can also be said of Internet of Things 
(IoT) systems, with  \textcite{gubbiInternetThingsIoT2013} stating that 'For the realization of a complete IoT vision, an efficient, secure,
scalable and market oriented computing and storage resourcing is essential.'

\para These efficient, secure and scalable resources exist in the form of NoSQL databases. Chapter 1 provided an overview of the various 
types of NoSQL databases, with this chapter instead offering direct comparisons between modern NoSQL solutions to more traditional relational 
database management systems.

\section{Foundational principles}
Relational and NoSQL databases are governed by dichotomous foundational principles that fundamentally change their appropriate use cases.
Relational databases adhere to the ACID model that prioritises absolute integrity and consistency at the expense of both availability and 
scalability, whereas NoSQL databases adhere to the opposing BASE model, prioritising availability and scalability over consistency.  

\subsection{ACID}
ACID is an acronym to describe the key attributes of relational database systems:

% TODO: Talk about ACID:
% ?     Atomicity - A statement is fully executed or not executed at all.
% ?     Consistency - Transactions only occur in predefined ways. (Helps prevent impacts of corruption)
% ?     Isolation - Concurrent transactions don't interfere with each other.
% ?     Durability - Changes are saved, even in the event of system failure. 
% *    https://www.databricks.com/glossary/acid-transactions

\begin{longtable}{ | p{0.2\textwidth} | p{0.7\textwidth} | }
    \hline
    \cellcolor{blue!25}Principle & \cellcolor{blue!25}Description\\
    \hline
    Atomicity & All transactions must be fully completed or not completed at all. If a transaction fails,
    the database must be reverted to its prior state with no changes made to the data.  \\
    \hline
    Consistency & Data must meet predefined integrity constraints, and remain consistent for all users even in the event 
    of simultaneous transactions.  \\
    \hline 
    Isolation & Transactions are executed sequentially, and do not interfere with each other even if they are simultaneous. \\
    \hline 
    Durability & Even in the event of system failure, the database must maintain all committed records. This means that 
    even if an error occurs, the results of any and all previous transactions must not be lost. \\
    \hline
    \caption{The ACID principles of relational databases. \autocite{awsACIDVsBASE,neo4jDataConsistencyModels2023}}\label{tab:ACID}
\end{longtable}

\noindent ACID is a very strict set of principles that actively enforce a safe environment for data operations. Though, in maintaining 
such high security, a considerable performance overhead exists with every transaction. This can massively impact the scalability of these 
databases, as they will grow slower the more users they have. 

\subsection{BASE and the CAP theorem}
% TODO: Talk about BASE:
% * A base is the opposite of an acid, hence the name.
% * Seems to be based on the *perception* of availability rather than the literal availability.
% ?     Basically Available - One transaction doesn't have to wait for another to complete first.
% ?     Soft State - The transitional state of records during concurrent updates. Finalised after all transactions complete.
% ?     Eventually Consistent - When all concurrent transactions are done, it will EVENTUALLY be consistent across all user perspectives.
% * https://aws.amazon.com/compare/the-difference-between-acid-and-base-database/

% TODO: Talk about the CAP theorem:
% ?     Consistency: Once data is written, it is available to all users of the system immediately.
% ?     Availability: The service is uninterrupted and without degradation for the majority of the time.
% ?     Partition Tolerance: Operations can be completed even if part of the network fails.
% * It's said that a distributed system can only ever achieve two of these.
% * https://www.ibm.com/think/topics/cap-theorem

\section{Scalability}


\subsection{Vertical}


\subsection{Horizontal}

% TODO: Talk about horizontal scalability.
% ?     Why is vertical bad? - It's much more expensive, and you'd probably have downtime during the actual upgrading.
% ?     Nodes and sharding

\section{Options for this scenario}

