\chapter{Comparing NoSQL and Relational databases} 
% ! Chapter title doesn't fit the page.
\begin{itemize}
    \item This section "shouldn't be too long, but enough to convince them."
    \item \textbf{This is NOT comparing \textit{MongoDB} to relational databases.} The brief 
    mentions that it should be "generic just in case after seeing your MongoDB database they decide to go with another software provider."
\end{itemize}

% TODO: Talk about ACID:
% ?     Atomicity - A statement is fully executed or not executed at all.
% ?     Consistency - Transactions only occur in predefined ways. (Helps prevent impacts of corruption)
% ?     Isolation - Concurrent transactions don't interfere with each other.
% ?     Durability - Changes are saved, even in the event of system failure. 
% *    https://www.databricks.com/glossary/acid-transactions

% TODO: Talk about BASE:
% * A base is the opposite of an acid, hence the name.
% * Seems to be based on the *perception* of availability rather than the literal availability.
% ?     Basically Available - One transaction doesn't have to wait for another to complete first.
% ?     Soft State - The transitional state of records during concurrent updates. Finalised after all transactions complete.
% ?     Eventually Consistent - When all concurrent transactions are done, it will EVENTUALLY be consistent across all user perspectives.
% * https://aws.amazon.com/compare/the-difference-between-acid-and-base-database/

% TODO: Talk about the CAP theorem:
% ?     Consistency: Once data is written, it is available to all users of the system immediately.
% ?     Availability: The service is uninterrupted and without degradation for the majority of the time.
% ?     Partition Tolerance: Operations can be completed even if part of the network fails.

% TODO: Talk about horizontal scalability.
% ?     Why is vertical bad?
% ?     Nodes and sharding

